\documentclass{article}
\usepackage{algorithm}
\usepackage{algorithmic}
\usepackage[a4paper, total={7in, 11in}]{geometry}
\usepackage{graphicx}
\usepackage[utf8]{inputenc}

\begin{document}
	\title{Laboratorio 3 - Juan Barillas}
	\maketitle
\section{Problema 1}
\section{Problema 2}
1. El running time cuando todos los valores son iguales seria de un $On^2$ debido a que sería parte de uno de los peores casos para quicksort. Realmente se debe cambiar también el psedocódigo para el partition.

2. 

3. Quicksort es mayormente utilizado pues a pesar de que tengan un peor worst-case running time, tiene un best-case y un average-case running time de $O(nlogn)$ lo cual lo hace más efectivo. Esto es debido a que nunca esperamos que tengamos el mejor de los casos para los algoritmos de ordenamiento.
\section{Problema 3}
	
	
	
	
\end{document}